\documentclass[11pt]{article}
\usepackage[utf8]{inputenc}	% Para caracteres en español
\usepackage{amsmath,amsthm,amsfonts,amssymb,amscd}
\usepackage{multirow,booktabs}
\usepackage[table]{xcolor}
\usepackage{fullpage}
\usepackage{lastpage}
\usepackage{enumitem}
\usepackage{fancyhdr}
\usepackage{mathrsfs}
\usepackage{wrapfig}
\usepackage{setspace}
\usepackage{calc}
\usepackage{multicol}
\usepackage{cancel}
\usepackage[margin=3cm]{geometry}
\usepackage{amsmath}
\newlength{\tabcont}
\setlength{\parindent}{0.0in}
\setlength{\parskip}{0.05in}
\usepackage{empheq}
\usepackage{framed}
\usepackage{xcolor}
\colorlet{shadecolor}{orange!15}
\parindent 0in
\parskip 12pt
\geometry{margin=1in, headsep=0.25in}
\theoremstyle{definition}
\newtheorem{defn}{Definition}
\newtheorem{reg}{Rule}
\newtheorem{exer}{Exercise}
\newtheorem{note}{Note}
\begin{document}
\setcounter{section}{0}
\title{Review Notes}

\thispagestyle{empty}

\begin{center}
{\LARGE \bf Summer Intern Lectures}\\
{\large Time Series Analysis Team}\\
June 2017
\end{center}
\section{State Space Models}
\subsection{What is a state space model?}
A state space model is a way to represent the dynamics of a set of variables. A models for observable variables $y_{i,t}$ can be thus described by a model of (possibly unobserved) $x_{i,t}$, i.e. the state variables. Any state space model can be rewritten as a system of an \textbf{observation equations}, relating the state space to the observation space, and a \textit{transformation equation}, describing the dynamics of the state space.


In its simplest form, a state space model can be characterized by the following system:

\begin{alignat}{2}
  y_{t} = G \cdot x_{t} \hspace*{5em} \text{observation equation} \\
  x_{t+1} = A\cdot x_{t} + C\cdot w_{t+1}, w_{t+1}\sim \mathcal{N}(0,I)  \hspace*{5em} \text{transition equation}
\end{alignat}

Note 1: this can be a higher-dimensional model as well, just make $y_{t}$ a $k\times 1$ vector and $x_{t}$ a $n\times 1$ vector. 

Then what do we have:

\begin{itemize}
\item $G$ is a $k\times n$ matrix known as the \textbf{output} matrix, relating state model to observable model.
\item $A$ is a $n\times n$ matrix known as the \textbf{transition} matrix, indicating the evolution of the state variable(s).
\item $C$ is a $\times m$ matrix known as the \textbf{volatility} matrix, indicating how uncertainty enters the model.
\item $w_{t+t1}$ is an $m\times 1$ vector of randomness.

Note 2: The state and observation models can have different dimensions! This will be very useful for \textit{Nowcasting} framework.

\end{itemize}

\begin{shaded}
\textbf{Example 1} \newline
The model for the variable $y_{t}$ is:

\begin{equation} y_{t+1} = \phi_{0} + \phi_{1} y_{t} + \phi_{2} y_{t-1} 
\end{equation}

 given $y_{0},y_{-1}$. How do you convert this into a state space model?

Look at what variables enter the model. It's a constant term as well as $y_{y}$ and $y_{t-1}$.
$$x_{t}:= [1\;\; y_{t}\;\; y_{t-1}]'.$$
Recall that the observation equation will be of the form $y_{t} = G x_{t}$. Given our state vector $x_{t}$ what is $G$? \newline

$G = [0\;\; 1\;\; 0]$ works! $$G\cdot x_{t} = \begin{pmatrix} 0 & 1 & 0 \end{pmatrix} \cdot \begin{pmatrix} 1\\ y_{t}\\ y_{t-1} \end{pmatrix} = y_{t}.$$

What are the dynamics of the state variable? We want to be able to reproduce the initial model (3) by relating $x_{t+1}$ and $x_{t}$ as in the transition equation: $x_{t+1} = Ax_{t} + C w_{t+1}$. What are $A$ and $C$?

$$x_{t+1} = \begin{pmatrix} 1\\ y_{t+1}\\ y_{t}\end{pmatrix} = A \begin{pmatrix} 1\\ y_{t}\\ y_{t-1} \end{pmatrix}$$

Solve analytically here, or just eye-ball it! $A=\begin{pmatrix} 1 & 0 & 0\\
\phi_{1} & \phi_{2} & \phi_{3}\\ 0 & 1 & 0 \end{pmatrix}$ works!

Finally, what is $C$? There is no randomness, so it $C=\vec{0}$.

\end{shaded}


For you to do:
\begin{itemize}
  \item[1)] $y_{t+1} = \phi_{1} y_{t} + \phi_{2} y_{t-1} + \phi_{3} y_{t-2} + \phi_{4} y_{t-3} + \sigma w_{t+1}$ with $w_{t} \sim \mathcal{N}(0,1)$.
  \item[2)] As above but $y_{t}$ is a $k\times 1$ vector (i.e. it represents $k$ many variables at time $t$). $\phi_{j}$ is $k\times k$ matrix and $w_{t}$ is $k\times 1$ vector. This is then a vector auto regression!
\end{itemize}

\end{document}