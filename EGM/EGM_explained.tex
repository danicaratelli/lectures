\documentclass[11pt]{article}
\usepackage[utf8]{inputenc}	% Para caracteres en español
\usepackage{amsmath,amsthm,amsfonts,amssymb,amscd}
\usepackage{multirow,booktabs}
\usepackage[table]{xcolor}
\usepackage{fullpage}
\usepackage{lastpage}
\usepackage{enumitem}
\usepackage{fancyhdr}
\usepackage{mathrsfs}
\usepackage{wrapfig}
\usepackage{setspace}
\usepackage{calc}
\usepackage{bm}
\usepackage{multicol}
\usepackage{cancel}
\usepackage[margin=3cm]{geometry}
\usepackage{amsmath}
\newlength{\tabcont}
\setlength{\parindent}{0.0in}
\setlength{\parskip}{0.05in}
\usepackage{empheq}
\usepackage{framed}
\usepackage{xcolor}
\colorlet{shadecolor}{orange!15}
\parindent 0in
\parskip 12pt
\geometry{margin=1in, headsep=0.25in}
\theoremstyle{definition}
\newtheorem{defn}{Definition}
\newtheorem{reg}{Rule}
\newtheorem{exer}{Exercise}
\newtheorem{note}{Note}
\begin{document}
\setcounter{section}{0}
\title{Review Notes}

\thispagestyle{empty}

\begin{center}
November 2018
\end{center}
\section{Endogenous Grid Method}
These notes are an adaptation of Josep Pijoan-Mas' notes with a few steps explained more in detail as well as the addition of some Julia code.

\subsection{Simple consumption/savings model}
Consider the dynamic programming problem
\begin{center}
	$v(a,y)= \underset{a',n}{\max\;}\left\{u(c,n) + \beta E_{y}\left[v(a',y')\right] \right\}$\\
	s.t. $a'+c\leq aR + nyw$,\\
	$a'\geq -b$, $c\geq 0$.
\end{center}
where the income process is a Markov chain, i.e. $y'$ only depends on $y$ and some stochastic factor. Substituting the (binding) budget constraint, gives us the FOCs are:
\begin{itemize}
	\item[$a')$] $$-u_{1}(c,n)+\beta E_{y}\left[v_{1}(a',n')\right]=0$$
	\item[$n)$] $$u_{1}(c,n)yw + u_{2}(c,n)=0$$
\end{itemize}
and the Envelope condition is:
\begin{itemize}
	\item[$a)$]$$v_{1}(a,y)=u_{1}(c,n)R.$$
\end{itemize}

Including the Envelope condition in the first FOC and assuming we can isolate $n$ from the second FOC gives us:
\begin{eqnarray}
u_{1}(c,n)=\beta R E_{y}\left[u_{1}(c',n')\right]\\
n = n(c,y)
\end{eqnarray}

\subsection{Euler Equation Iteration}
Here follow the steps for the Euler Equation iteration.
\begin{enumerate}
	\item Guess a policy function for consumption $\bm{c=g_{0}^{c}(a,y)}$. This will be updated and will determine our convergence.
	\item Substitute this rule on the right-hand side of (1) and get:
	$$u_{1}(c)=\beta R E_{y}\left[u_{1}(g_{0}^{c}(a',y'),n')\right],$$
	Assuming $u_{1}(\cdot)$ is invertible, invert the equation just derived in order to get consumption:
	$$c = u_{1}^{-1}\left( \beta R E_{y}\left[u_{1}(g_{0}^{c}(a',y'),n')\right] \right)=:\tilde{g}_{0}^{c}(a',y)$$
	Notice that $\tilde{g}_{0}^{c}(\cdot)$ depends on $a'$ because, in the above expression those are known at at $0$ and it depends on $y$ because the expectation is taken over income and next period income only depends on current income.\\
	This formulation tells us that consumption today is a function of assets tomorrow and income today. So, knowing $y$ and $a'$, we can pin down $c$.
	\item If the asset grid is $A =\{a_{1},\dots a_{m}\}$ and the income process is discrete, i.e. $Y=\{y_{1},\dots,y_{n}\}$ and Markovian (as we are assuming) with transition matrix $\Gamma$, we can write
	$$c =\tilde{g}_{0}^{c}(a_{i},y_{j})= u_{1}^{-1}\left( \beta R \sum_{l} \Gamma_{j,l} u_{1}(g_{0}^{c}(a_{i},y_{l}),n')\right)$$
	Notice further that $A$ is the asset grid for tomorrow's assets.\\
	
	Next, using the (binding) budget constraint we can deduce current assets given tomorrow's assets and current income, namely
	\begin{align*}
	a_{i} + c &= a_{i,j} R + ny_{j}w\\
	\Rightarrow a_{i,j}^{\ast}&=\frac{a_{i}+c-ny_{j}w}{R}
	\end{align*}
	If hours worked were exogenous we would be done, however, we can, making some separability assumptions, say as before that $n=n(c,y_{j})=n(\tilde{g}_{c}^{0}(a_{i},y_{j}),y_{j})=:\tilde{g}_{0}^{n}(a_{i},y_{j})$ and so
	\begin{equation}
	a_{i,j}^{\ast}=\frac{a_{i}+\tilde{g}_{c}^{0}(a_{i},y_{j})-\tilde{g}_{0}^{n}(a_{i},y_{j})y_{j}w}{R}
	\end{equation}
	Eq. (3) gives us a set of new asset grids $A^{\ast}_{j}=\{a_{1,j}^{\ast},\dots a_{n,j}^{\ast}\}$, one for each possible income state. Furthermore, we have consumption at each one of these asset positions, or in other words, the new policy function for the new asset grid is $g_{c}^{1}(a_{i,j}^{\ast},y_{j})=\tilde{g}_{c}^{0}(a_{i},y_{j})$.
	\item  Next we need to map the new policy function, defined on the grid $A_{j}^{\ast}$, onto the old asset grid $A$, i.e. we need to find $g_{c}^{1}(a_{i},y_{j})$. We do so as follows:
	\begin{enumerate}
		\item If $a_{i}\leq a_{1,j}^{\ast}$ then assets next period are very low which means that this period the agent will be at the borrowing constraint, i.e. $a_{i}=a_{1}$ and so, from the budget constraint we get:
		$$g_{1}^{c}(a_{i},y_{j})=a_{i}R+ny_{j}w-a_{1}=a_{i}R+g_{1}^{n}(a_{i},y_{j})y_{j}w-a_{1}.$$
		\item If $a_{i}>a_{1,j}^{\ast}$ on the other hand $a_{i}$ must lie within some $\left[a_{k,j}^{\ast},a_{k+1,j}^{\ast}\right]$ interval. Simply interpolate $g_{1}^{c}(a_{i,j},y_{j})$ from $g_{1}^{c}(a_{k,j},y_{j})$ and $g_{1}^{c}(a_{k+1,j},y_{j})$.
	\end{enumerate}
	\item Finally, the updated policy function is:
	\begin{equation}
	g_{c}^{1}(a_{i},y_{j})=\begin{cases}
	a_{i}R+ny_{j}w-a_{1}=a_{i}R+g_{1}^{n}(a_{i},y_{j})y_{j}w-a_{1} \;\; \text{if } a_{i}\leq a_{i,j}^{\ast}\\
	interp\left(g_{1}^{c}(a_{k,j},y_{j}),g_{1}^{c}(a_{k+1,j},y_{j})\right) \;\; \text{if } a_{i}> a_{i,j}^{\ast}
	\end{cases}
	\end{equation}
\end{enumerate}


\section{Julia example}
\begin{shaded}
Here is the example...
\end{shaded}


\end{document}